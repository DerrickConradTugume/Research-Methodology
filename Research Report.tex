\documentclass{report}

\begin{document}
\title{REPORT ON THE INSTALLATION OF RYTROCK ADMINISTRATION SYSTEM}
\author{Tugume Derrick Conrad\\13/U/14328/PS}
\date{\today}
\maketitle

\section {INTRODUCTION}

The workers of ETM International Church have suffered a severe problem of losing records and ineffective activity management with the recent expansion of the church’s premises and increasing projects. Rytrock Administration System has been implemented successfully to address this problem.

\section {BACKGROUND}
Six months ago, Mr. Kuteesa John, an administrator at the church informed me that he would like to have an administration system not only to monitor and supervise the workers but also to safely keep their records and activities. Observing what took place for a week, the workers would write their reports, activities and minutes and submit them hand written to the administrator. This was time consuming and costly because they would purchase the papers and spend time moving from their departments to submit their data. To solve the problem, I together with Namugenyi Angel and Kiyega Denis developed and implemented Rytrock Administration System at the church.

\section {SYSTEM IMPLEMETATION}
We used ASP.NET, a Microsoft Visual Studio platform for developing web systems to implement the code. In addition to that, we used mySQL server to develop and manage the database, HTML (Hypertext Transfer Protocol), JavaScript and CSS (Cascading Style Sheet) to develop the system interfaces. 
We developed the first prototype from 05th October 2016 to 21st January 2017, deployed the prototype on the intranet server at the church premises on 22nd January 2017. Some of the challenges we faced while developing the prototype include insufficient funds to cater for miscellaneous costs like transportation fees and airtime, inaccurate information provided by the workers which would lead to inaccuracy of the system and unavailability of some workers when needed especially the departmental workers. 
Despite the above mentioned challenges, we managed to develop and deploy the prototype on the server. We then used the incremental model approach to add other system requirements as were requested by the workers. However, we faced technical challenges at the server during deployment like having a lower dot net framework version which was solved by upgrading the server software. We frequently visited the church premises to get feedback from the workers on how the new system was affecting them. From the media department, they said that the system was fastening the rate at which they submit their reports since they no longer had to walk to the administrator’s office but only submit their data through the system.
\\
\\
Although the system was running correctly after a series of tests and adjustments, the IT department requested us to make it much more secure as they revealed the security loops the system had. This forced us to shutdown the system for two weeks in order to address the issue. Later, we redeployed the system and availed for testing for a period of one month.
After fully testing the system with improved security, the IT department approved the system for use by the church. Currently, the workers are using the system to store and submit their files, reports, activities and minutes. In appreciation for developing and deployment of the system, the church rewarded us with Ushs 500,000.
\section {RECOMMENDATIONS}
The church should equip the workers with computer literacy knowledge to make it easier for the workers to easily and comfortably interact with the system modules.
The church should also purchase more computers to give system access to all the departments since some departments like canteen and Peace Center do not have a dedicated computer given to them.
The IT department should frequently manage the intranet server and troubleshoot it in times when it is either hacked or shutdown unexpectedly. In addition to this, the IT team should implement more security techniques to secure the network and the system.
\\\\
Thank you. If you have any questions, please contact me at tugumeconrad@gmail.com or 0778 418 592.
\end{document}